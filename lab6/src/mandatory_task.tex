\begin{multicols}{2}

\begin{center}
    \textbf{ЕГЭ ПО ФИЗИКЕ}\\[14pt]
    \textit{\fontsize{11}{12}\selectfont Вариант 1}
\end{center}

\vspace{-0.8em}
\textbf{Ответы к заданиям 1--26}
\vspace{0.5em}

\begingroup
\footnotesize
\setlength{\tabcolsep}{2pt}
\renewcommand{\arraystretch}{1.3}
\centering
\noindent
\resizebox{\linewidth}{!}{%
\begin{tabular}{|c|c|c|c|c|c|c|c|c|c|}
\hline
\textbf{1} & \textbf{2} & \textbf{3} & \textbf{4} & \textbf{5} & \textbf{6} & \textbf{7} & \textbf{8} & \textbf{9} & \textbf{10} \\ \hline
5 & 8 & 0,4 & 0,04 & 34 или 43 & 21 & 21 & 3 & 60 & 4000 \\ \hline
\end{tabular}
}
\vspace{2pt}
\noindent
\resizebox{\linewidth}{!}{%
\begin{tabular}{|c|c|c|c|c|c|c|}
\hline
\textbf{11} & \textbf{12} & \textbf{13} & \textbf{14} & \textbf{15} & \textbf{16} & \textbf{17} \\ \hline
45 или 54 & 23 & \parbox{1.5cm}{\centering к наблю-\\дателю} & 1 & 4 & 13 или 31 & 21 \\ \hline
\end{tabular}
}
\vspace{2pt}
\noindent
\resizebox{\linewidth}{!}{%
\begin{tabular}{|c|c|c|c|c|c|c|c|c|}
\hline
\textbf{18} & \textbf{19} & \textbf{20} & \textbf{21} & \textbf{22} & \textbf{23} & \textbf{24} & \textbf{25} & \textbf{26} \\ \hline
13 & 3694 & 0,3 & 23 & 0,200,02 & 14 или 41 & 500 & 5 & 5 \\ \hline
\end{tabular}
}
\endgroup

\vspace{1.0em}

\noindent\textbf{Задание 27.} При поднесении отрицательно заряженной палочки свободные заряды проводника
перераспределяются так, чтобы потенциал всех точек проводника остался одинаковым (явление электростатической индукции).
В данном случае отрицательные свободные заряды (электроны) переместятся с левого электрометра на правый.
Правый электрометр зарядится отрицательно (избыток электронов), левый -- положительно (недостаток электронов).

Если бы мы удалили заряженную палочку не убирая стержня, то свободные заряды вернулись бы на прежнее место и
электрометры стали бы вновь не заряженными.
Но поскольку мы сначала убрали стержень, то при удалении заряженной
палочки заряды не смогли переместиться с электрометра на электрометр.
В конечном состоянии правый
электрометр окажется заряженным отрицательно, а левый -- положительно.

\noindent\textbf{Задание 28.} На высоте $h$ на кубик действуют две силы: сила тяжести, равная $mg$, и сила реакции опоры, численно равная силе давления кубика $F$.
Из второго закона Ньютона в проекции на радиальное направление (ось $x$; рис.\,8)
\begin{equation*}
    F + mg \cos \alpha = m \frac{v^2}{R},
\end{equation*}
где $\cos \alpha = (h - R)/R$, выразим $v^2$ и подставим в закон сохранения энергии
\begin{equation*}
    mgH = mgh + \frac{mv^2}{2}.
\end{equation*}

% --- КАРТИНКА ---
\vspace{0.5em}
\noindent
\begin{center}
    \includegraphics[width=0.95\linewidth, height=10cm, keepaspectratio]{../docs/figure}
\end{center}
\vspace{0.5em}

\columnbreak

\noindent Получим
\begin{equation*}
    H = \frac{3h - R}{2} + \frac{FR}{2mg} = 3,25 \text{ м} \cdot
\end{equation*}

\noindent\textbf{Задание 29.} Объем и давление столбика воздуха в начальном состоянии равны
\begin{equation*}
    V_1 = S l_1, \quad p_1 = p_0 = \rho_{\text{рт}} g h_0 ,
\end{equation*}
где $S$ -- площадь внутреннего сечения трубки, $h_0 = 747 \text{ мм} = 74,7 \text{ см}$.
Объем и давление в конечном состоянии равны
\begin{equation*}
    V_2 = S l_2, \quad p_2 = p_0 + \rho_{\text{рт}} g l = \rho_{\text{рт}} g (h_0 + l) ,
\end{equation*}
где $l$ -- искомая длина ртутного столбика.
Подставив эти выражения в уравнение изотермического процесса
\begin{equation*}
    p_1 V_1 = p_2 V_2 ,
\end{equation*}
получим равенство
\begin{equation*}
    l_1 h_0 = l_2 (h_0 + l) ,
\end{equation*}
откуда найдем
\begin{equation*}
    l = h_0 \left( \frac{l_1}{l_2} - 1 \right) \approx 21,7 \text{ см} \cdot
\end{equation*}

\noindent\textbf{Задание 30.} В первом стержне возникает ЭДС индукции $\mathscr{E}_1 = B v_1 l$, во втором -- $\mathscr{E}_2 = B v_2 l$.
Ток в контуре определяется законом Ома для полной цепи:
\begin{equation*}
    I = \frac{\mathscr{E}_1 - \mathscr{E}_2}{2R} = \frac{B l v_{\text{отн}}}{2R} \cdot
\end{equation*}
На первый стержень действует сила Ампера, равная $F_{\text{A}} = I B l$ и направленная назад, на второй стержень действует такая же сила, но направленная вперед.
Кроме того, на каждый стержень действует направленная назад сила трения $F_{\text{тр}} = \mu mg$.
Второй закон Ньютона для первого и второго стержней имеет вид
\begin{equation*}
    F - IBl - \mu mg = 0, \quad IBl - \mu mg = 0 \cdot
\end{equation*}
Сложив эти уравнения, получим $F = 2\mu mg$ -- только при таком значении внешней силы возможно равномерное движение стержней.
Тогда из уравнения движения второго стержня и выражения для силы тока получим
\begin{equation*}
    \frac{B^2 l^2 v_{\text{отн}}}{2R} = \mu mg ,
\end{equation*}
откуда найдем
\begin{equation*}
    v_{\text{отн}} = \frac{2 \mu mg R}{B^2 l^2} = 2 \text{ м/с} \cdot
\end{equation*}

\noindent\textbf{Задание 31.} Сила тока перестает зависеть от напряжения, когда анода достигают все электроны, выбитые светом с катода.
Через ток насыщения выражается число электронов, выбитых с катода за 1 с:
\begin{equation*}
    N_{\text{эл}} = \frac{I_{\text{нас}}}{e} \cdot
\end{equation*}

\end{multicols}
